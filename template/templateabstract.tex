%---------------------------------------------------------------------
% ABSTRACT CONFIGURATION
%---------------------------------------------------------------------

%---------------------------------------------------------------------
%   01: PACKAGES
\usepackage[utf8]{inputenc}
\usepackage[italian]{babel}
\usepackage{xspace}
\usepackage{epsfig,amsmath,amsthm,amsfonts,graphicx,amssymb}
\usepackage{times}

\usepackage{lipsum}

%\linespread{1.0}
%---------------------------------------------------------------------
%   02: PAGE LAYOUT
% HORIZONTAL POSITIONING >>>>>>>>>>>>>>>>>>>>>>>>>>>>>>>>>>>>>>>>>>>>>
\setlength{\hoffset}         { -1in}  % 1 right margin offset = zero
\addtolength{\hoffset}       { 00mm}  % 1 crop to 17cm size: flushing right (including 10mm ligature)
\setlength{\oddsidemargin}   { 20mm}  % 3  left/odd  margin of +32mm
\setlength{\evensidemargin}  { 20mm}  % 3* left/even margin of +16mm
\setlength{\textwidth}       {170mm}  % 8 width  of the text of 130mm
\setlength{\marginparsep}    {  0mm}  % 9 margin notes separation
\setlength{\marginparwidth}  {  0mm}  %10 margin notes width
% TOTAL...................... 210mm OK!
% VERTICAL  POSITIONING>>>>>>>>>>>>>>>>>>>>>>>>>>>>>>>>>>>>>>>>>>>>>>>>
\setlength{\voffset}       {   -1in}  % 2 top margin offset to balance +1in offset
\addtolength{\voffset}     { 20.0mm}  % 2 center in the page: 43.5 mm
\setlength{\topmargin}     {  0.0mm}  % 4 vertical margin of 0 mm    %
\setlength{\headheight}    {  0.0mm}  % 5 header height of 5 mm      % voffset + 10mm
\setlength{\headsep}       {  0.0mm}  % 6 header separation of 5 mm  %
\setlength{\textheight}    {257.0mm}  % 7 length of the text of 190mm
\setlength{\footskip}      { 05.0mm}  %11 footer height + separation 10mm (=5mm+5mm)
% TOTAL.................... 297mm OK!
%----------------------------------------------------------------------


\usepackage{ifthen}

\newboolean{tesiAbilitaDedica}
\newboolean{tesiAbilitaRingraziamenti}


%% Comandi vuoti per il template
\newcommand{\tesiUniversita}{\empty}
\newcommand{\tesiDipartimento}{\empty}
\newcommand{\tesiCdL}{\empty}

\newcommand{\tesiCandidato}{\empty}
\newcommand{\tesiMatricola}{\empty}
\newcommand{\tesiTitolo}{\empty}
\newcommand{\tesiAA}{\empty}
\newcommand{\tesiData}{\empty}

\newcommand{\tesiRelatoreA}{\empty}

%% BEGIN: Info Template Tesi
\newcommand{\defUniversita}[1]{\renewcommand{\tesiUniversita}{#1}}
\newcommand{\defDipartimento}[1]{\renewcommand{\tesiDipartimento}{#1}}
\newcommand{\defCdL}[1]{\renewcommand{\tesiCdL}{#1}}

\newcommand{\defCandidato}[1]{\renewcommand{\tesiCandidato}{#1}}
\newcommand{\defMatricola}[1]{\renewcommand{\tesiMatricola}{#1}}
\newcommand{\defTitolo}[1]{\renewcommand{\tesiTitolo}{#1}}
\newcommand{\defAA}[1]{\renewcommand{\tesiAA}{#1}}
\newcommand{\defData}[1]{\renewcommand{\tesiData}{#1}}

\newcommand{\defRelatoreA}[1]{\renewcommand{\tesiRelatoreA}{#1}}
\newcommand{\defRelatoreB}[1]{\newcommand{\tesiRelatoreB}{#1}}
\newcommand{\defRelatoreC}[1]{\newcommand{\tesiRelatoreC}{#1}}
% END: Info Template Tesi

\title{\tesiTitolo}
\author{
    Candidato: \emph{\tesiCandidato} (mat. \emph{\tesiMatricola}) \\
    Relatore: \tesiRelatoreA; \\ 
    \ifthenelse{\equal{\tesiRelatoreB}{\empty}}{}{
        Correlatore: \tesiRelatoreB 
        \ifthenelse{\equal{\tesiRelatoreC}{\empty}}{}{, \tesiRelatoreC}
    \\}
    Seduta di laurea del \tesiData 
}
\date{\emph{Abstract}}